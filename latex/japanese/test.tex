\title{人工知能課題8}
\author{機械情報工学科3年 120292 若田部亮}
\date{\today}

\documentclass{jarticle}
\begin{document}

\maketitle

\section{概要}
この資料は幾何の自動証明に用いられるWuの定理のプログラム課題の説明,及びレポートである.
今回,数式処理プログラムを使用することなく,多項式の四則演算及び剰余の計算は全て自分で実装した. 

\section{プログラム説明}
それぞれのファイルの役割を記述する.
\begin{itemize}
    \item monopoly.cpp 単項式の実装
    \item poly.cpp 多項式の定義
    \item polyinfrastructure.cpp 多項式の環境の定義
    \item wumethod.cpp 三角化及びWuの手続きの実装
    \item main.cpp Wuの手続きに入力するための幾何問題を記述
\end{itemize}

実行方法は
{\small
\begin{verbatim}
make
./a.out
\end{verbatim}
}
による.

\section{実行結果}
出力結果を以下に貼り付ける.
{\small
\begin{verbatim}
/************************* WuMethod ***************************/
/*** Wu 1 ***/
Vars: x y 
Hyp: y-x -1.5 -1.5 = 0
Hyp: x\^2  + 2 y = 0
Conc: 4 y = 0
Tri: y-x -1.5 -1.5 = 0
Tri: -y + y\^2 + 2.25 2.25 = 0
failed for unexpected reminder at the end.
4 y
Tri: y-x -1.5 -1.5 = 0
Tri: x\^2  + 2 x  + 3 3 = 0
failed for unexpected reminder at the end.
4 x  + 6 6
failed triangulation.
Tri: x\^2  + 2 y = 0
Tri: -x -1.5 -1.5-0.5 x\^2  = 0
failed for unexpected reminder at the end.
4 x  + 6 6
FALSE
/*** Wu 2 ***/
Vars: x y 
Hyp: 3 x y\^2 = 0
Hyp: 2 x\^2 y = 0
Conc: 4 y = 0
failed triangulation.
failed triangulation.
failed triangulation.
failed triangulation.
FALSE
/*** Wu 3 ***/
Vars: x y 
Hyp: 3 x y = 0
Hyp: 2 x\^2 y = 0
Conc: 4 y = 0
failed for 0 poly in tripoly.
failed for 0 poly in tripoly.
failed triangulation.
failed triangulation.
FALSE
/*** Wu 4 ***/
Vars: x y 
Hyp: x\^2  + 2 y = 0
Hyp: 3 x  = 0
Conc: 4 y = 0
failed triangulation.
Tri: x\^2  + 2 y = 0
Tri: 3 x  = 0
TRUE
/*** Wu 5 ***/
Vars: x y 
Hyp: x  + y  + z = 0
Hyp: x  + y  = 0
Conc: 4 z = 0
Tri: x  + y  + z = 0
Tri: -z = 0
TRUE
/*** Wu 7-1 ***/
Vars: a b 
Hyp: e-2 a  = 0
Hyp: 2 a -b -d  = 0
Conc: -2 a b  + b\^2  + 2 a d -d\^2  = 0
Tri: e-2 a  = 0
Tri: -b -d  + e = 0
TRUE
/*** Wu 7-2 ***/
Vars: a b 
Hyp: e-2 a  = 0
Hyp: d -b  = 0
Conc: -2 a b  + b\^2  + 2 a d -d\^2  = 0
Tri: e-2 a  = 0
Tri: d -b  = 0
TRUE
\end{verbatim}
}

実際にはここに書かれているものよりも上に出力があるが,これは多項式演算のテスト関数であるため,課題には無関係である.
この出力のVarsは証明に際して有効な独立変数の列挙であり,Hypは前提条件を示し,Concが示したい対象を表し,Triが三角化された前提条件となっている.
TRUEまたはFALSEが真偽となるが,FALSEは正確には偽ではなく,「わからない」ことを意味している.

Wu 7-1は高校受験数学掲示板から探してきた幾何の問題を解いた.

http://www.inter-edu.com/forum/read.php?908,359105

\fbox{
Mは線分ABの中点、AB//BD、CM=DM⇒AC=BCを証明せよ。 
}

この問題に対して,点Aを原点, 点B方向をx軸とする.点A = (0, 0), 点M = (a, 0), 点B = (e, 0), 点C = (b, c), 点D = (d, c)と置いて,CM = DMの条件と加えて書き下すと,
前提条件と証明すべき式を列挙すると,\\
\\
中点条件: $Hyp_1: e = 2 a$\\
距離一致条件: $Hyp_2: = (a - b)^2 - (a - d)^2 = (2 a - b - d) (d - b)$\\
証明すべき式: $b^2 + c^2 = (e - d)^2 + c^2$\\
\\
今,距離一致条件が因数分解可能な形となっている.因数分解可能な場合は予め因数分解をして,その条件を2つに分解し,その両方で証明できなければならない.従って,この問題を解くためには,以下の2つの問題を両方解く必要がある.\\
\\
問題1\\
\\
中点条件: $Hyp_1: e = 2 a$\\
距離一致条件: $Hyp_2: = 2 a - b - d$\\
証明すべき式: $b^2 + c^2 = (e - d)^2 + c^2$\\
\\
問題2\\
\\
中点条件: $Hyp_1: e = 2 a$\\
距離一致条件: $Hyp_2: = d - b$\\
証明すべき式: $b^2 + c^2 = (e - d)^2 + c^2$\\
\\
出力のWu 7-1が上の問題1に相当し,Wu 7-2が問題2に相当している.そしてこの結果,両方でTRUEであるため,本問題は真であることがわかる.

\section{感想}
今回Wuの定理を実装するにあたって資料が極めて少なく,一般に報告されているWuの定理の性能をはるかに下回るものしか作ることが出来なかった.
Wuの定理のReminderなどの記法に戸惑い,理解に相当苦しんだが,何とか雰囲気から理解した.
しかし,この課題を提出した今でも,多項式の三角化に関する知識は不十分である.
多項式の三角化に関する資料はグレブナー基底のものを除いて,一切見つからなかった.
そのため今回の課題では,パーミュテーションで前提条件の関数を並べて,貪欲に前提条件同士を割っていった結果,うまく三角化されてくれたらそれを使い,そうでなければFALSEを返す,という乱暴な処理を行なってしまっている.

今回の課題のコード数を数えてみたところ,全てあわせて2668行,Wuの定理に関わる部分に限ると342行とかなり多項式演算クラスに時間が取られてしまった.
おとなしく数式処理ソフトにパイプを繋いでいれば,と後悔するばかりである.

\end{document}
